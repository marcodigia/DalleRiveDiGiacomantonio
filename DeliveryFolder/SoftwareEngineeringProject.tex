\documentclass{article}

\usepackage{graphicx}

\renewcommand{\labelenumii}{\theenumii}
\renewcommand{\theenumii}{\theenumi.\arabic{enumii}.}
\renewcommand{\theenumiii}{\theenumii\arabic{enumiii}.}


\begin{document}
\begin{figure}[t!]
	\includegraphics[width= \linewidth]{PolimiLogo.png}
	\begin{center}
	Politecnico di Milano\\[4pt]
	AA 2018-2019  \\[4pt]
	Computer Science and Engineering \\[4pt]
	\begin{large}
	Software Engineering 2 Project
	\end{large}
	\end{center}
\end{figure}
\begin{flushright}
\begin{large}
Dalle Rive Fabio - 920082 \\[4pt]
Di Giacomantonio Marco -
\end{large}
\end{flushright}
\newpage
\textbf{Table of Contents}
	\begin{enumerate}
			\item Introduction
			\begin{enumerate}
				\item Purpose
				\item Scope
				\begin{enumerate}
					\item Description of the given problem
					\item Goals
				\end{enumerate}
				\item Definitions, Acronyms, Abbreviations
				\begin{enumerate}
					\item Definitions
					\item Acronyms
					\item Abbreviations
				\end{enumerate}
				\item Document structure
			\end{enumerate}
			\item Overall Description
			\begin{enumerate}
				\item Product perspective
				\item Product functions
				\begin{enumerate}
					\item 
				\end{enumerate}
				\item User characteristics
				\item Assumptions, dependencies and constraints
				\begin{enumerate}
					\item Domain assumptions
				\end{enumerate}
			\end{enumerate}
			\item Specific requirements
			\begin{enumerate}
				\item External interface requirements	
				\begin{enumerate}
					\item Users interfaces
					\item Hardware interfaces
					\item Software interfaces
					\item Communication interfaces
				\end{enumerate}
				\item Scenarios
				\begin{enumerate}
					\item
				\end{enumerate}
				\item Functional requirements
				\begin{enumerate}
					\item Use case diagram
					\item Sequence diagram
				\end{enumerate}
				\item Performance requirements
				\item Design Constraints
				\begin{enumerate}
					\item Standard compliance
					\item Hardware limitation
					\item Other constraint
				\end{enumerate}
				\newpage
				\item Software system attributes
				\begin{enumerate}
					\item Reliability
					\item Security
					\item Maintainability
					\item Compatibility
				\end{enumerate}
			\end{enumerate}
			\item Formal Analysis Using Alloy
			\begin{enumerate}
				\item Alloy model
				\item World generated
				\item Alloy results
			\end{enumerate}
			\item Effort Spent
			\item Resources
	\end{enumerate}
	\newpage
\section{Introduction}
\subsection{Purpose}
The purpose of this project is to build a system, call it Data4Help, that allows third parties to monitor the position and health status of users. The data are collected by TrackMe, the company that wants to develop Data4Help, and are shared with other companies which are interested in those data. AutomatedSOS is a system build on top of Data4Help. AutomatedSOS is a service designed for elderly people which is able to intervene calling an ambulance, if the health parameters of the user are below some fixed thresholds.
\subsection{Scope}
\subsubsection{Description of the given problem}
TrackMe is a company that wants to develop a software-based service allowing third parties to monitor the location and health status of users. This service is called Data4Help. The service supports the registration of the visitors who, by registering, allows TrackMe to acquire their data. Also it supports the registration of third parties. After registration, these third parties can request: 
\begin{itemize}
	\item Access to the data of some specific user.
	\item Access to anonymized data of groups of users.
\end{itemize}
TrackMe also wants to develop a non-intrusive SOS service for elderly people, call it AutomatedSOS. AutomatedSOS is build on top of Data4Help. This service is designed to monitor health status of users and send, to the location of the user, an ambulance if some parameters are below specified thresholds. 
\subsubsection{Goals}
\begin{itemize}
	\item {[G1]} Visitor can become a registered user after providing credentials.
	\item {[G2]} User can be recognized providing a username and a password.
	\item {[G3]} User can accept or reject the request of access to his data formulated by companies.
	\item {[G4]} If user's parameters are below specified thresholds, an ambulance is called within 5 seconds. 
	\begin{itemize}
		\item {[G4.1]} Ambulance is required at current user's location. 
	\end{itemize}
	\item {[G5]} Company can become a registered user after providing credentials. 
	\item {[G6]} Company can be recognized providing a username and a password.
	\item {[G7]} Company can formulate a request to see anonymized data of a group of users.
	\item {[G8]} Company can formulate a request to see data of a specific user providing his SSN.
	\item {[G9]} Company can see anonymized data of a group of users.
	\item {[G10]} Company can see data of a specific user providing his SSN.
	\item {[G11]} Company can subscribe to users' new data.
	\item {[G12]} TrackMe can anonymized data.
	\item {[G13]} TrackMe can show users' requested data to companies.
	\item {[G14]} TrackMe can forward companies' requests to users. 
	\item {[G15]} TrackMe can sign a user as follow-able  by a specific company. 
	\item {[G16]} ??? TrackMe able to withdraw a comapny ????
\end{itemize} 
\subsection{Definitions, Acronyms, Abbreviations}
\subsubsection{Definitions}
\begin{itemize}
	\item\textbf{Visitor:} a person who still have to register to Data4Help and AutomatedSOS.
	\item\textbf{User:} a person who is registered to Data4Help and/or AutomatedSOS.
	\item\textbf{Third Companies:} 
	\item\textbf{Data:} Location of a user + heart rate + calories burned + time spent exercising + step walked 
	\item \textbf{Threshold:} Flexible value related to the biomedical data that can be acquired by the wearable device in which the system is installed.  This value is computed by well-known equation that operate with user's data. It also depends from the kind of activity that a user is doing.  
	\item \textbf{Request:} Formal request that a company issue to TrackMe in order to access data of a single user or a group of users. 
	\item \textbf{Subscribe to data:} A company which is subscribed to user's data or to group's data, receives the requested data as soon as they are produced.  
	\item \textbf{Pendent user:} A user that a company want to follow but has not answered to the following request yet.
\end{itemize}
\subsubsection{Acronyms}
\begin{itemize}
	\item RASD - Required Analysis and Specification Document
	\item GPS - Global Positioning System
	\item SSN - Social Security Number
\end{itemize}
\subsubsection{Abbreviations}
\begin{itemize}
	\item {[Gn]}: n-th goal
	\item {[Dn]}: n-th domain assumption
	\item {[Rn]}: n-th functional requirement
\end{itemize}
\subsection{Document structure}
\begin{description}
	\item [Introduction] gives an introduction to the problem and describe the purpose of Data4Help and AutomatedSOS. It also contains the goals that these systems must be able to deliver to users and third party companies.
	\item [Overall Description]
	\item [Specific Requirements]
	\item [Formal Analysis Using Alloy] includes the Alloy model and the discussion of its purpose. Also, a world generated by this model is shown.
	\item [Effort Spent] shows the effort spent by each group member while working on this project.
	\item [Resources] includes the reference documents. 
\end{description}
\section{Overall Description}
\subsection{Product perspective}
\subsection{Product functions}
\subsection{User characteristics}
\subsection{Assumption, dependencies and constraints}
\subsubsection{Domain assumptions}
\begin{itemize}
	\item {[D1]} The user's email is already known by TrackMe.
	\item {[D2]} The user's registration process is carried out through a survey that is sent to the user by an external survey service. 
	\item {[D3]} The user
	\item {[D4]}
	\item {[D5]}
	\item {[D6]}
	\item {[D7]}
	\item {[D8]}
	\item {[D9]}
	\item {[D10]}
	\item {[D11]}
\end{itemize}
\section{Specific Requirements}
\subsection{External interface requirements}
\subsubsection{User interfaces}
\subsubsection{Hardware interfaces}
\subsubsection{Software interfaces}
\subsubsection{Communication interfaces}
\subsection{Scenarios}
\subsection{Functional requirements}
\subsubsection{Use case diagram}

\begin{center}
    \begin{tabular}{ | l | p{10cm} |}
    \hline
    Name & Visitor sing up.\\ \hline
    Actors & User\\ \hline
   	Goals & {[G1]}\\ \hline
    Input Conditions & There are no entry conditions.\\ \hline
    Event Flow & \begin{enumerate}
    	\item The user on the home page clicks on the sign in button to start the registration process.
    	\item The user fills all the mandatory fills.
		\item The user clicks on the confirm button.
		\item The system saves the data.
    \end{enumerate} \\ \hline
    Output Conditions & The user successfully ends the registration process. From now on he can log into the application, providing his credentials and start using Data4Help. \\ \hline
    Exceptions & \begin{enumerate}
    	\item The user is already registered.
		\item The user inserts no valid information in one or more mandatory fills.
		\item The user chooses a username that has already been taken. 
		\item The user chooses an email that his associated with another user.
	\end{enumerate}
All the exceptions are handled notifying the issues to the user and taking back the event flow to point 2.
    \\ \hline
    \end{tabular}
\end{center}

\begin{center}
    \begin{tabular}{ | l | p{10cm} |}
    \hline
    Name & User login.\\ \hline
    Actors & User\\ \hline
   	Goals & {[G2]}\\ \hline
    Input Conditions & There user is already in the homepage.\\ \hline
    Event Flow & \begin{enumerate}
    	\item The user inserts is credentials in to the username and password fills.
		\item The user clicks on the login button in order to access.
		\item The system redirects the user to his profile.
    \end{enumerate} \\ \hline
    Output Conditions & The user his successfully redirects to his profile. \\ \hline
    Exceptions & \begin{enumerate}
    	\item The user insert a not valid username.
		\item The user insert a not valid password.
	\end{enumerate}
All the exceptions are handled notifying the issues to the user and taking back the event flow to point 1.
    \\ \hline
    \end{tabular}
\end{center}

\begin{center}
    \begin{tabular}{ | l | p{10cm} |}
    \hline
    Name & User accept or reject third party requests.\\ \hline
    Actors & User\\ \hline
   	Goals & {[G3]}\\ \hline
    Input Conditions & There user is already logged.\\ \hline
    Event Flow & \begin{enumerate}
    	\item The user picks up a request.
		\item The user decides whether accepting it or not.
		\item The request is withdrawn from the list of requests. 
		\item The user is successfully redirect to his home page. 
    \end{enumerate} \\ \hline
    Output Conditions & The user is notified to have actually accepted/reject the request. The third part is notified that a request has been accepted/rejected. \\ \hline
    Exceptions & - \\ \hline
    \end{tabular}
\end{center}

\begin{center}
    \begin{tabular}{ | l | p{10cm} |}
    \hline
    Name & Third part sign up.\\ \hline
    Actors & Third part\\ \hline
   	Goals & {[G5]}\\ \hline
    Input Conditions & There are no entry conditions.\\ \hline
    Event Flow & \begin{enumerate}
    	\item The third part on the home page clicks on the sign in button to start the registration process.
		\item The third part fills all the mandatory fills.
		\item The third part clicks on the confirm button.
		\item The system saves the data.
    \end{enumerate} \\ \hline
    Output Conditions & The third part successfully ends the registration process. From now on it can log into the application, providing his credentials and start using Data4Help.  \\ \hline
    Exceptions & \begin{enumerate}
    \item The third part is already registered.
	\item The third part inserters no valid informations in one or more mandatory fills.
	\item The third part chooses a username that has already been taken. 
	\item The third part chooses an email that his associated with another third part.
\end{enumerate} All the exceptions are handled notifying the issues to the third part and taking back the event flow to point 2.  \\ \hline
    \end{tabular}
\end{center}

\begin{center}
    \begin{tabular}{ | l | p{10cm} |}
    \hline
    Name & Third part login.\\ \hline
    Actors & Third part\\ \hline
   	Goals & {[G6]}\\ \hline
    Input Conditions & There third part is already in the homepage.\\ \hline
    Event Flow & \begin{enumerate}
    	\item The third part inserts is credentials in to the username and password fills.
		\item The third part clicks on the login button in order to access.
		\item The system redirects the third part to his profile.
    \end{enumerate} \\ \hline
    Output Conditions & The third part his successfully redirects to his profile.  \\ \hline
    Exceptions & \begin{enumerate}
   \item The third part insert a not valide username.
	\item The third part insert a not valid password.
\end{enumerate} All the exceptions are handled notifying the issues to the user and taking back the event flow to point 1.    \\ \hline
    \end{tabular}
\end{center}

\begin{center}
    \begin{tabular}{ | l | p{10cm} |}
    \hline
    Name & Third part formulate a request to access anonymized data of a group of users.\\ \hline
    Actors & Third part\\ \hline
   	Goals & {[G7]}\\ \hline
    Input Conditions & The third part is already logged in into the system.\\ \hline
    Event Flow & \begin{enumerate}
    	\item The third part selects the form for a group request.
    	\item The request is filled using drop-down menu. Each drop-down menu is linked to a type of filter. By selecting a filter the company is able to better specify the composition of the group it is interested in.
		\item The request is sent to TrackMe by clicking on Send button. 
    \end{enumerate} \\ \hline
    Output Conditions & The request is sent to TrackMe and a message for the correct sending of the request is presented to the company. \\ \hline
    Exceptions & -   \\ \hline
    \end{tabular}
\end{center}

\begin{center}
    \begin{tabular}{ | l | p{10cm} |}
    \hline
    Name & Third part formulate a request to access data of a specific user through is SSN. \\ \hline
    Actors & Third part\\ \hline
   	Goals & {[G8]}\\ \hline
    Input Conditions & The third part is already logged in into the system.\\ \hline
    Event Flow & \begin{enumerate}
    	\item The third part selects the form for a specific user request.
    	\item The request is filled with the user's SSN.
    	\item The request is sent to TrackMe by clicking on Send button. 
    \end{enumerate} \\ \hline
    Output Conditions & The request is sent to TrackMe and a message for the correct sending of the request is presented to the company.  \\ \hline
    Exceptions & -    \\ \hline
    \end{tabular}
\end{center}

\begin{center}
    \begin{tabular}{ | l | p{10cm} |}
    \hline
    Name & Third part can access anonymized data of a group of users.\\ \hline
    Actors & Third part\\ \hline
   	Goals & {[G9]}\\ \hline
    Input Conditions & There third part is already logged into the system.\\ \hline
    Event Flow & \begin{enumerate}
    	\item The third part accesses the approved group requests section.
		\item The third part selects an approved group request.
		\item The system redirects the third part to the request result.
    \end{enumerate} \\ \hline
    Output Conditions & The data requested by the third part are shown.  \\ \hline
    Exceptions & \begin{enumerate}
  		\item The data requested aren't anonymous since the outcome of the request concerns less than 1000 users.
  		\item The request was not approved by more than 1000 users belonging to the specific group.
\end{enumerate} All the exceptions are handled notifying the issues to the third part and taking back the event flow to point 1.    \\ \hline
    \end{tabular}
\end{center}

\begin{center}
    \begin{tabular}{ | l | p{10cm} |}
    \hline
    Name & Third part can access data of a specific user through his SSN.\\ \hline
    Actors & Third part\\ \hline
   	Goals & {[G10]}\\ \hline
    Input Conditions & There third part is already logged into the system.\\ \hline
    Event Flow & \begin{enumerate}
    	\item The third part accesses the approved single user requests section.
    	\item The third part selects an approved single user request. 
    	\item The system redirects the third part to the request result.
    \end{enumerate} \\ \hline
    Output Conditions & The data requested by the third part are shown.  \\ \hline
    Exceptions & \begin{enumerate}
   \item The user didn't approve the request.
\end{enumerate} All the exceptions are handled notifying the issues to the third part and taking back the event flow to point 1.    \\ \hline
    \end{tabular}
\end{center}

\begin{center}
    \begin{tabular}{ | l | p{10cm} |}
    \hline
    Name & TrackMe can anonymise data. \\ \hline
    Actors & TrackMe\\ \hline
   	Goals & {[G12]}\\ \hline
    Input Conditions & Third part has requested data of a group of users.\\ \hline
    Event Flow & \begin{enumerate}
    	\item The third part has requested data of a group of users.
		\item TrackMe performs the research among the matching users.
		\item TrackMe anonymize the data of the selected users.
    \end{enumerate} \\ \hline
    Output Conditions & The data requested from the third part are anonymized. \\ \hline
    Exceptions & \begin{enumerate}
  		\item The data requested aren’t anonymous since the outcome of the request concerns less than 1000 users.
  		\item The request was not approved by more than 1000 users belonging to the specific group.
\end{enumerate} All the exceptions are handled notifying the issues to the third part and taking back the event flow to point 1.    \\ \hline
    \end{tabular}
\end{center}

\begin{center}
    \begin{tabular}{ | l | p{10cm} |}
    \hline
    Name & TrackMe can show anonymized data related to a group of users to third parties. \\ \hline
    Actors & TrackMe\\ \hline
   	Goals & {[G13]}\\ \hline
    Input Conditions & Third part has requested data of a group of users OR Third part has requested data of a single user.\\ \hline
    Event Flow & TrackMe sends the result of the research to the third part. \\ \hline
    Output Conditions & The data requested from the third part are sent. \\ \hline
    Exceptions & -  \\ \hline
    \end{tabular}
\end{center}

\begin{center}
    \begin{tabular}{ | l | p{10cm} |}
    \hline
    Name & TrackMe can forward third parties requests to users. \\ \hline
    Actors & TrackMe\\ \hline
   	Goals & {[G14]}\\ \hline
    Input Conditions & Third part has requested data of a group of users OR Third part has requested data of a single user.\\ \hline
    Event Flow & \begin{enumerate}
    	\item TrackMe sends forward the request of being observed by a specific company to the user.
    	\item TrackMe marks the user as pendent user.
    \end{enumerate} \\ \hline
    Output Conditions & The request of being followed by a company is sent to the user. \\ \hline
    Exceptions & -    \\ \hline
    \end{tabular}
\end{center}

\begin{center}
    \begin{tabular}{ | l | p{10cm} |}
    \hline
    Name & TrackMe can sign a user as follow-able. \\ \hline
    Actors & TrackMe\\ \hline
   	Goals & {[G15]}\\ \hline
    Input Conditions & TrackMe is notified that a user has answered to a following request.\\ \hline
    Event Flow & \begin{enumerate}
    	\item TrackMe accesses the profile relative to the specific user.
    	\item TrackMe reset the state of the user. The user is not pendent anymore.
    \end{enumerate} \\ \hline
    Output Conditions & The company that issued the request is notified \\ \hline
    Exceptions & TrackMe receives a notification by a user that is not pendent. The notification is ignored by TrackMe.\\ \hline
    \end{tabular}
\end{center}

\subsubsection{Sequence diagram}
\subsection{Performance requirements}
\subsection{Design constraints}
\subsubsection{Standard compliance}
\subsubsection{Hardware limitation}
\subsubsection{Other constraints}
\subsection{Software system attributes}
\subsubsection{Reliability}
\subsubsection{Security}
\subsubsection{Maintainability}
\subsubsection{Compatibility}
\section{Formal Analysis Using Alloy}
\subsection{Alloy model}
\subsection{World generated}
\subsection{Alloy results}
\section{Effort Spent}
\section{Resources}
\end{document}


